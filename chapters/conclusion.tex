%%==================================================
%% conclusion.tex for BIT Master Thesis
%% modified by yang yating
%% version: 0.1
%% last update: Dec 25th, 2016

%% modified by Meng Chao
%% version: 0.2
%% last update: May 29th, 2017
%%==================================================


\begin{conclusion}
为解决无人机自主飞行中的导航和定位问题,本文围绕单目视觉SLAM算法进行研究,从算法和功能上对原有基于特征的单目SLAM进行改进和拓展。特别的,借鉴基于直接法的SLAM重构和多传感器融合领域的成果,对基于特征的单目SLAM算法进行改进和完善,取得了较好的效果。本文主要研究成果如下:
\begin{enumerate}  [label={(\arabic*)}]
\item 介绍多旋翼无人机导航定位算法发展,详细综述视觉SLAM的流程和核心算法。对视觉SLAM的理论和算法发展进行了总结,对比单目、双目和RGB-D三种SLAM算法特点应用场景。结合国内外视觉SLAM的发展现状和发展趋势,本文认为,就算法角度来看,单目视觉SLAM算法未来的发展方向是半稠密以上的环境地图重建,提供丰富的环境信息;从应用层面出发,单目SLAM可与IMU、GPS和激光雷达等多传感器进行数据融合,提供准确的尺度信息,改善单目SLAM鲁棒性。本文的改进工作主要围绕以上两点展开。
\item 针对特定的使用对象,研究多旋翼无人机的数学模型和控制率设计。完成无人机的动力学和运动学建模,并对非线性部分进行线性简化。根据经典控制理论,使用串级PID控制器控制无人机的位置和姿态。通过MATLAB仿真,验证无人机数学模型的准确性和控制率的可行性,了解其运动特性,便于后续选择适合无人机的单目SLAM算法。
\item 详细介绍了单目视觉SLAM的结构框架、算法原理和分类。从核心原理、定位精度、鲁棒性和地图重构4个方面系统的比较了基于直接法的LSD-SLAM和基于特征的ORB-SLAM。相比于LSD-SLAM,基于特征的ORB-SLAM具有较好的视角不变性和光度不变性,可以在宽基线下稳定匹配,对于快速运动具有较好的鲁棒性,适宜作为无人机的定位感知系统。但ORB-SLAM也存在一些问题,如重构地图稀疏,无法用于避障和路径规划;单目相机无法获取深度信息,运动轨迹和地图尺度不确定。
\item 在第四章中,针对基于特征的ORB-SLAM算法地图稀疏的问题,研究一种基于特征的单目半稠密SLAM算法。该算法参考直接法SLAM重构原理,采用像素块匹配和基于概率的逆深度假设进行半稠密重构。针对宽基线下的像素块匹配重构地图点云逆深度离群值较多的问题,引入逆深度一致性检剔除离群值,提高半稠密地图重构效果。实验表明改进后的基于特征的单目半稠密SLAM算法定位精度高,重构的环境半稠密地图可以满足避障与路径规划的需要。
\item 在第五章中,就单目SLAM算法尺度不确定的问题,研究基于IMU预积分的惯性-视觉SLAM算法,在传统基于特征的SLAM基础上,通过预积分算法对IMU数据进行处理,将IMU测量数据表示为最大后验观测模型的形式,通过SLAM的后端非线性优化与
IMU进行数据融合。实验表明,基于IMU预积分的惯性-视觉SLAM算法可以准确估计运动轨迹的尺度和传感器偏移,整个算法定位精度高;IMU先验运动信息还可加速特征跟踪和匹配,提高算法鲁棒性。
\end{enumerate}

针对以上援救成果和结论,本文对应用于多旋翼无人机导航定位的单目SLAM算法进行了深入研究与全面分析。但由于时间有限,能力欠缺,仍存在一些不足,后续工作可针对以下几个方面深入展开:
\begin{enumerate}  [label={(\arabic*)}]
\item 本文研究的单目半稠密重建只使用CPU处理,因而相机图片质量和帧率均受到限制。可考虑采用GPU加速的方法进行三维重构,可提高处理运算效率,增加地图稠密程度。
\item 当前针对惯性-视觉SLAM的IMU初始化没有公认的判定方法确定IMU初始化的准确性,后续可考虑结合线性系统的条件数研究自动的判定准则,判断IMU初始化的状态变量准确性。
\end{enumerate}





\end{conclusion}