%%==================================================
%% conclusion.tex for BIT Master Thesis
%% modified by yang yating
%% version: 0.1
%% last update: Dec 25th, 2016

%% modified by Meng Chao
%% version: 0.2
%% last update: May 29th, 2017
%%==================================================


\chapter*{总结与展望\markboth{总结与展望}{}}
\addcontentsline{toc}{chapter}{总结与展望}

We accomplish a monocular SLAM which is called LSD-SLAM using a novel direct method, which is tested in real-time on a CPU. It maintains and tracks on global map of environment, which is different from existing approach such as feature approach. It contains a pose graph of key frames with associated probabilistic semi-dense depth map. Major components of the LSD-SLAM are two feature. First, a direct method to align two key frames on sim(3), explicitly incorporating and detecting scale-drift. Second, a novel probabilistic approach in [10]to incorporate noise on the estimated depth maps. Our experiment showed that the LSD-SLAM reliably tracks, trajectories and maps for hand-held. We think that this method can be used in unmanned aerial vehicles(UAV) for navigation and flight control. The LSD-SLAM demonstrate the versatility, robustness and flexibility.



\iffalse
本文采用……。{\color{blue}(结论作为学位论文正文的最后部分单独排写,但不加章号。结论是对整个论文主要结果的总结。在结论中应明确指出本研究的创新点,对其应用前景和社会、经济价值等加以预测和评价,并指出今后进一步在本研究方向进行研究工作的展望与设想。结论部分的撰写应简明扼要,突出创新性。)本文采用……。(结论作为学位论文正文的最后部分单独排写,但不加章号。结论是对整个论文主要结果的总结。在结论中应明确指出本研究的创新点,对其应用前景和社会、经济价值等加以预测和评价,并指出今后进一步在本研究方向进行研究工作的展望与设想。结论部分的撰写应简明扼要,突出创新性。)}
\fi