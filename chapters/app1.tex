%%==================================================
%% app1.tex for BIT Master Thesis
%% modified by yang yating
%% version: 0.1
%% last update: Dec 25th, 2016

%% modified by Meng Chao
%% version: 0.2
%% last update: May 29th, 2017
%%==================================================


\chapter{IMU预积分中的数学原理}

\section{李群与李代数性质}

李群中的特殊正交群$SO(3) \doteq \{ \boldsymbol{R} \in \mathds{R}^{3 \times 3}:\ \boldsymbol{R}^T \boldsymbol{R}=\boldsymbol{I},det(\boldsymbol{R})=\boldsymbol{I} \}$表示刚体在三维空间的旋转矩阵,李群$SO(3)$在流型空间上单位元素处的正切为李代数$\mathfrak{so}(3) \doteq \{ \boldsymbol{\phi}: \boldsymbol{\phi} \in  \mathds{R}^3  \}$。
李代数可以通过指数映射$\exp(\cdot)$与李群相关联,有
\begin{equation}
\label{equ1}
\boldsymbol{R} = \exp \left( \boldsymbol{\phi}^\wedge \right) = \boldsymbol{I}+{ \sin( \Vert \boldsymbol{\phi} \Vert) \over \Vert \boldsymbol{\phi} \Vert } \boldsymbol{\phi}^{\wedge} + { {1-\cos(\Vert \boldsymbol{\phi} \Vert)} \over \Vert \boldsymbol{\phi} \Vert^2 }\left(\boldsymbol{\phi}^{\wedge}\right)^2
\end{equation}
其中$(\cdot)^\wedge$表示向量的反对称矩阵,反对称矩阵具有如下性质。
\begin{equation}
\boldsymbol{a}^\wedge \boldsymbol{b} = - \boldsymbol{b}^\wedge \boldsymbol{a} \ \ \ \forall \boldsymbol{a},\boldsymbol{b} \in \mathds{R}^3
\end{equation}
李群也可通过对数映射$\log(\cdot)$与李代数相关联,设$\boldsymbol{\phi} = \boldsymbol{a} \varphi$
\begin{equation}
\label{equ2}
\begin{aligned}
\boldsymbol{\phi} &= \log(\boldsymbol{R}) = { \varphi \cdot \left(\boldsymbol{R} - \boldsymbol{R}^T \right)  \over 2 \sin(\varphi) }
\\
\varphi &= \cos^{-1} \left( tr(\boldsymbol{R})-1 \over 2 \right)
\end{aligned}
\end{equation}
李代数到李群的指数映射泰勒展开的一阶近似可以表示为
\begin{equation}
\label{equ3}
\begin{aligned}
\exp\left( \boldsymbol{\phi}^{\wedge} \right) & \approx  \boldsymbol{I} + \boldsymbol{\phi}^{\wedge}
\\
\exp\left( \boldsymbol{\phi}^{\wedge} + \delta\boldsymbol{\phi}^{\wedge}  \right) & \approx \exp \left( \boldsymbol{\phi}^{\wedge} \right)  \exp \left( J_r\left( \boldsymbol{\phi} \right) \delta\boldsymbol{\phi}^{\wedge} \right)
\end{aligned}
\end{equation}
其中$J_r\left( \boldsymbol{\phi} \right)$表示$SO(3)$相对于李代数微小增量的右雅各比矩阵
\begin{equation}
\label{equ4}
J_r\left( \boldsymbol{\phi} \right) = \boldsymbol{I} - {{1-\cos(\Vert \boldsymbol{\phi} \Vert)} \over \Vert \boldsymbol{\phi} \Vert^2 } \boldsymbol{\phi}^{\wedge}+
{{ \Vert \boldsymbol{\phi} \Vert-\sin(\Vert \boldsymbol{\phi} \Vert)} \over \Vert \boldsymbol{\phi^3} \Vert }\left(\boldsymbol{\phi}^{\wedge}\right)^2
\end{equation}
李代数指数映射与李群的乘积有以下性质
\begin{equation}
\label{equ5}
\begin{aligned}
\boldsymbol{R} \exp \left( \boldsymbol{\phi} \right) \boldsymbol{R}^T &= \exp \left( \boldsymbol{R}  \boldsymbol{\phi}^{\wedge} \boldsymbol{R}^T \right) = \exp \left(  \boldsymbol{R} \boldsymbol{\phi} \right)
\\ 
\Leftrightarrow \ \ \ \exp \left( \boldsymbol{\phi} \right) \boldsymbol{R} &=  \boldsymbol{R} \exp \left(\left( \boldsymbol{R}^T  \boldsymbol{\phi} \right)^\wedge\right)
\end{aligned}
\end{equation}

