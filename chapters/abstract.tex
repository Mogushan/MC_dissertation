%%==================================================
%% abstract.tex for BIT Master Thesis
%% modified by yang yating
%% version: 0.1
%% last update: Dec 25th, 2016

%% modified by Meng Chao
%% version: 0.2
%% last update: May 29th, 2017
%%==================================================

\begin{abstract}

近年来无人机发展迅速,传统的经典导航方式存在应用场景和定位精度的局限,无法满足当前无人机对导航定位的需要。基于视觉的同步定位与地图重建(SLAM)可以在运动的过程中,同时完成定位与环境地图重构。单目相机作为视觉SLAM常用传感器,其结构简单、计算效率高,适用于无人机飞行的大尺度和变尺度环境。本文围绕基于单目视觉的SLAM算法,研究适用于多旋翼无人机的导航定位系统。主要研究内容如下:

首先,研究多旋翼无人机的数学模型,依据线性化假设对无人机运动学和动力学模型进行化简,通过数学仿真了解无人机的运动特性,便于后续分析和选择选用于无人机导航定位的视觉SLAM算法。

其次,研究单目SLAM结构框架、算法原理和分类。从定位精度、鲁棒性和地图重构效果三个方面比较了基于直接法的LSD-SLAM和基于特征的ORB-SLAM。实验结果表明,相比于LSD-SLAM算法,基于特征的ORB-SLAM定位精度高、鲁棒性好,适宜作为无人机导航定位系统。但同时,ORB-SLAM存在一些问题,如重构环境地图稀疏,无法用于避障和路径规划;单目相机无法获取深度信息,估计的轨迹和地图尺度不确定。

然后,针对基于特征的ORB-SLAM算法重构地图稀疏的问题,研究一种基于特征的单目半稠密SLAM算法,参考基于直接法SLAM的重构原理,采用像素块匹配和逆深度假设恢复环境的半稠密地图。针对基于特征的SLAM算法在宽基线下像素块匹配离群值较多的问题,引入逆深度一致性检验剔除离群值,提高半稠密地图重构效果。实验表明,基于特征的单目半稠密SLAM算法定位精度高,鲁棒性好,恢复的环境半稠密地图可以满足无人机避障与路径规划的需要。

最后,针对单目SLAM尺度不确定的问题,研究基于IMU预积分的惯性-视觉SLAM算法。通过预积分对IMU数据进行处理,得到适用于最大后验估计的IMU观测模型,通过视觉SLAM后端的非线性优化与IMU进行数据融合。实验表明,基于IMU预积分的惯性-视觉SLAM算法可以准确估计运动轨迹尺度,整个算法定位精度高,鲁棒性好。


\keywords{无人机;SLAM;单目视觉;半稠密;多传感器融合}


\end{abstract}




\begin{englishabstract}

  
\englishkeywords{UAV; SLAM; monocular; semi-dense; multi-sensor fusion}

\end{englishabstract}
