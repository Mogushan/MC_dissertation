%%==================================================
%% abstract.tex for BIT Master Thesis
%% modified by yang yating
%% version: 0.1
%% last update: Dec 25th, 2016

%% modified by yang yating
%% version: 0.1
%% last update: Dec 25th, 2016
%%==================================================

\begin{abstract}
 
本文……。({\color{blue}{摘要是一篇具有独立性和完整性的短文,应概括而扼要地反映出本论文的主要内容。包括研究目的、研究方法、研究结果和结论等,特别要突出研究结果和结论。中文摘要力求语言精炼准确,硕士学位论文摘要建议500$\sim$800字,博士学位论文建议1000$\sim$1200字。摘要中不可出现参考文献、图、表、化学结构式、非公知公用的符号和术语。英文摘要与中文摘要的内容应一致。}})

\keywords{形状记忆;聚氨酯;织物;合成;应用 ({\color{blue}{一般选3~8个单词或专业术语,且中英文关键词必须对应。})}}

\end{abstract}

\begin{englishabstract}

   In this papers, we accomplish simultaneous localization and mapping using the monocular LSD-SLAM. This method is different with feature method, estimating the accurate and reconstructing the large scale environment map. Using the direct image alignment, the environment can be mapped in pose-graph of key frames semi-dense maps. The LSD-SLAM contains two advantage. A novel direct tracking method can definitely estimate the scale-drift. In addition, the probabilistic estimation can optimize the effect of noisy depth values on tracking. The experiment showed the LSD-SLAM is reliability, robustness and real-time.
   
\englishkeywords{direct image alignment; LSD-SLAM; probabilistic estimation; semi-dense maps}

\end{englishabstract}
