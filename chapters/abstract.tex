%%==================================================
%% abstract.tex for BIT Master Thesis
%% modified by yang yating
%% version: 0.1
%% last update: Dec 25th, 2016

%% modified by Meng Chao
%% version: 0.2
%% last update: May 29th, 2017
%%==================================================

\begin{abstract}

近年来随着无人机的普及与发展,由于传统的经典导航方式存在应用场景和定位精度的局限,无法满足无人机对导航和定位的需要。基于视觉的同步定位与地图重建(SLAM)算法可以在运动的过程中,同时完成定位与环境地图重构。单目相机作为视觉SLAM常用传感器,其结构简单、计算效率高,适用于大尺度和变尺度的环境。本文围绕基于单目视觉的SLAM算法,研究适用于多旋翼无人机的视觉SLAM导航定位算法。主要研究内容如下:

首先研究多旋翼无人机的数学模型,依据线性化假设对无人机运动学和动力学模型进行化简,通过数学仿真了解无人机的运动特性,便于后续分析和选择适合无人机的视觉SLAM算法。

其次,研究视觉单目SLAM结构框架、算法原理和分类。从原理、定位精度、鲁棒性和地图重构效果4个方面比较了基于直接的LSD-SLAM和基于特征的ORB-SLAM。实验结果表明,相比于LSD-SLAM,基于特征的ORB-SLAM定位精度高、鲁棒性好,适宜作为无人机视觉导航定位系统。但同时,ORB-SLAM存在一些问题,如重构环境地图稀疏,无法用于避障和路径规划;由于单目相机无法获取深度信息,估计的轨迹和地图尺度不确定。

然后,针对基于特征的ORB-SLAM算法重构地图稀疏的问题,研究一种基于特征大的单目半稠密SLAM算法,该算法参考基于直接法SLAM的重构原理,采用像素块匹配和逆深度假设恢复环境的半稠密地图。针对基于特征的SLAM算法在宽基线下像素块匹配离群值较多的问题,引入逆深度一致性检验剔除离群值,提高半稠密地图重构效果。实验表明,改进的基于特征的单目半稠密SLAM算法定位精度高,鲁棒性好,恢复的环境半稠密地图可以满足避障与路径规划的需要。

最后,针对单目SLAM尺度不确定的问题,研究基于IMU预积分的惯性-视觉SLAM算法。通过预积分算法对IMU数据进行处理,获取IMU数据基于最大后验估计的观测模型,通过视觉SLAM后端的非线性优化算法与IMU数据进行融合。实验表明,基于IMU预积分的惯性-视觉SLAM算法可以准确估计运动轨迹尺度。整个算法定位精度高,鲁棒性好。


\keywords{无人机;SLAM;单目视觉;半稠密;多传感器融合}





\iffalse
本文……。({\color{blue}{摘要是一篇具有独立性和完整性的短文,应概括而扼要地反映出本论文的主要内容。包括研究目的、研究方法、研究结果和结论等,特别要突出研究结果和结论。中文摘要力求语言精炼准确,硕士学位论文摘要建议500$\sim$800字,博士学位论文建议1000$\sim$1200字。摘要中不可出现参考文献、图、表、化学结构式、非公知公用的符号和术语。英文摘要与中文摘要的内容应一致。}})

\keywords{形状记忆;聚氨酯;织物;合成;应用 ({\color{blue}{一般选3~8个单词或专业术语,且中英文关键词必须对应。})}}

\fi

\end{abstract}



\begin{englishabstract}

  
\englishkeywords{direct image alignment; LSD-SLAM; probabilistic estimation; semi-dense maps}

\end{englishabstract}
