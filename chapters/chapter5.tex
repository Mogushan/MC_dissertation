%%==================================================
%% chapter02.tex for BIT Master Thesis
%% modified by yang yating
%% version: 0.1
%% last update: Dec 25th, 2016

%% modified by Meng Chao
%% version: 0.2
%% last update: May 29th, 2017
%%==================================================
\chapter{基于IMU预积分的视觉惯性单目SLAM算法}
\label{chap:VISLAM}


%5.1
\section{IMU预积分算法}
The traditional feature-based approaches, both filtering-based and key frame-based, is to split the overall problem estimating geometric information from images into two sequential steps. Firstly, a series of feature observations is recognized from the image. Then the camera position, posture and scene geometry is computed as a function of these feature observations only. Splitting simplifies the overall problem, it comes with an important limitation. Only information that conforms to the feature type can be used. Kind of method is proposed in [3] such as edge-based or even region-based features.

Direct visual odometry (VO) methods solve this limitation by optimizing the geometry directly on the image intensities,which use all information in the image. While direct image alignment is well-established for RGB-D or stereo sensors [4], only recently monocular direct VO algorithms have been proposed. Accurate and fully dense depth maps are computed using a variational formula which however is computationally demanding and requires a state-of-the-art GPU to run in real-time [5]. By combining direct tracking with key points, it can achieve high framerates even on embedded platform [6].


%5.2
\section{视觉惯性单目SLAM算法状态初始化}
It is a well-know and grown approach to optimize the estimate of the camera posture and position and improve the performance of the global map. This method considers that the world is represented as a number of key frames connected by pose-pose constraints, which can be optimized using a generic graph optimization framework like g2o.


%5.3
\section{视觉惯性单目SLAM算法状态估计与优化}
A 3D rigid body transform $G{\rm{ }} \in {\rm{ }}SE(3)$  denotes rotation and translation in 3D,  i.e. is defined by

\begin{equation}
G=\left(
    \begin{array}{cc}
      R & t \\
      0 & 1 \\
    \end{array}
  \right)
    \ with
    \ R\in SO(3)
    \ and
    \ t\in {\mathbb{R}}^{3}
\end{equation}
SE(3) and SO(3) is the concept of elements of Lie-Algebras.

We define the 3D projective warp function $\omega $, which projects an image point ${\bf{p}}$ and its inverse depth ${\bf{d}}$ into a by $\xi $ transformed camera frame.

\begin{equation}
\begin{split}
\omega(p,d,\xi):= \left(
                     \begin{array}{c}
                       x^{'}/z^{'} \\
                       y^{'}/z^{'} \\
                       1/z^{'} \\
                     \end{array}
                   \right) \ with \\
                   \left(
                     \begin{array}{c}
                       x^{'} \\
                       y^{'} \\
                       z^{'} \\
                       1 \\
                     \end{array}
                   \right):=\!exp_{\!se(\!3\!)}(\xi)
                   \left(
                     \begin{array}{c}
                       p_{x}/d \\
                       p_{y}/d  \\
                       1/d  \\
                       1 \\
                     \end{array}
                   \right)
\end{split}
\end{equation}

A 3D similarity transform   denotes rotation, scaling and translation, i.e. is defined by

\begin{equation}
S=\left(
    \begin{array}{cc}
      sR & t \\
      0 & 1 \\
    \end{array}
  \right)
  \ with
  \ R\in SO(3),t\in {\mathbb{R}}^{3}
  \ and \ s\in {\mathbb{R}}^{+}
\end{equation}

As for rigid body transformations, a minimal representation is given by elements of the associated Lie-algebra $\xi  \in sim(3)$, which now have an additional degree of freedom, that is $\xi\in {\mathbb{R}}^{7}$.



\section{本章小结}





